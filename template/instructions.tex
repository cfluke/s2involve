%% template.tex
 %
 % Copyright 2006-2012 David G. Barnes, Paul Bourke, Christopher Fluke
 %
 % This file is part of S2PLOT.
 %
 % S2PLOT is free software: you can redistribute it and/or modify it
 % under the terms of the GNU General Public License as published by
 % the Free Software Foundation, either version 3 of the License, or
 % (at your option) any later version.
 %
 % S2PLOT is distributed in the hope that it will be useful, but
 % WITHOUT ANY WARRANTY; without even the implied warranty of
 % MERCHANTABILITY or FITNESS FOR A PARTICULAR PURPOSE.  See the GNU
 % General Public License for more details.
 %
 % You should have received a copy of the GNU General Public License
 % along with S2PLOT.  If not, see <http://www.gnu.org/licenses/>. 
 %
 % We would appreciate it if research outcomes using S2PLOT would
 % provide the following acknowledgement:
 %
 % "Three-dimensional visualisation was conducted with the S2PLOT
 % progamming library"
 %
 % and a reference to
 %
 % D.G.Barnes, C.J.Fluke, P.D.Bourke & O.T.Parry, 2006, Publications
 % of the Astronomical Society of Australia, 23(2), 82-93.
 %
 % $Id: s2direct.tex,v 1.7 2011/05/19 00:50:36 dbarnes Exp $
 %

\documentclass{article}
\usepackage[margin=1.5cm]{geometry}
\usepackage{graphicx}
\usepackage{hyperref}
\usepackage[3D]{movie15}
\parindent=0pt
\parskip=1.2\baselineskip

\begin{document}

\begin{figure}
\begin{center}
  \vspace{0.12in}
  \includemovie[label=fig:OBJNAME,
    poster=OBJNAME-poster.png,
    3Dcoo=0.007216811180114746 -0.04447293281555176 -0.0332721471786499,
    3Droo=6.881564140319824,
    3Dviews2=s2views.txt,
    3Djscript=s2plot-prc.js,
    controls,
    3Dlights=CAD,
    3Dbg=0 0 0,
    autoplay,
    3Drender=Solid,
  ]{WIDTH\linewidth}{HEIGHT\linewidth}{s2direct.prc}\\
\end{center}
\caption{OBJNAME. Three-dimensional visualisation performed with the 
{\sc S2PLOT} programming library (Barnes et al. 2006), using {\sc s2involve} 
(Fluke \& Barnes, {\em in prep}).  This is an interactive figure that you can 
view with Adobe Reader (Version 8.0 or higher).
\movieref[3Dgetview]{fig:OBJNAME}{Set default view}
}
\label{caption:fig1}
\end{figure}

\newpage
\section*{Instructions}
PRC files exported from S2PLOT (\url{http://astronomy.swin.edu.au/s2plot}, 
version 3.2 or higher) can be embedded as interactive 3-d figures in PDF documents 
prepared with {\sc pdflatex}.  


All components required for recreating this PDF document are in the DIR-OBJNAME
directory.  Running the {\sc generatepdf.csh} script again will remove any files, including
those you may already have edited. 

\subsection*{Setting a default view}
To select the default view that will appear when your interactive figure is viewed
with Adobe Reader (version 8.0 or higher):
\begin{enumerate}
\item You will need to change into the DIR-OBJNAME directory to edit files and
run {\tt pdflatex}.
\item Modify the camera angle, choose any specific lighting or background colour 
such that you have the desired default 3D view. 
\item Click the label ``Set default view'' in the caption of Figure \ref{caption:fig1}.
\item In the pop-up window that opens, you will see text something like:\\

\begin{verbatim}
VIEW%={<insert descriptive name here (optional)>}
  C2C=0 0 1
  ROO=4.949749973032196
  BGCOLOR=0.000000 0.000000 0.000000
  LIGHTS=CAD
  RENDERMODE=Solid
END
\end{verbatim}

\item Copy this text (you may omit any {\tt PART} descriptions - but be sure that 
your {\tt VIEW} has an {\tt END}) and paste it into the file {\tt s2views.txt}.  
Provide a useful descriptive name, e.g. {\tt Default}, remembering to remove the \% 
after {\tt VIEW}.  You can refer to the {\tt S2PLOT} standard views for further
examples of the correct format. 

\item Exit Adobe Reader.

\item Execute the commands:
\begin{verbatim}
% pdflatex publish-OBJNAME.tex
% pdflatex instruction-OBJNAME.tex
\end{verbatim}
to update the interactive figures.
\end{enumerate}

When you now view with Adobe Reader, the first view in this {\tt s2views.txt} file
will be the default configuration when the figure opens. Click on the
interactive figure to access the 3D menu, and then you can select this, and
other views, from the Views menu.

%Once you have finished modifying your views, remember to delete or comment out
%the caption line 
%\begin{verbatim}
%\movieref[3Dgetview]{fig:OBJNAME}{Set default view}
%\end{verbatim}
%and then execute the command:
%%\begin{verbatim}
%%% pdflatex publish-OBJNAME.tex
%%\end{verbatim}
to update the interactive figure.
 
{\bf Caution:} Any changes you make to the publish-OBJNAME.tex or {\tt s2views.txt} 
files will be lost if you re-run {\tt generatepdf.csh OBJNAME}.


\section*{Credits}
If you find this approach to visualising spectral data cubes to be useful, 
it would be greatly appreciated if you can cite our work as follows:
{\em Three-dimensional visualisation performed with the {\tt S2PLOT} programming 
library (Barnes et al. 2006), using {\sc s2involve} (Fluke \& Barnes, {\em in prep}).}

Additional references may be made to our original description of the 3D-PDF approach
for astronomy publications (Barnes \& Fluke 2008) and the free, open-source solution 
using the PRC format in \LaTeX (Barnes et al. 2013). 


\begin{thebibliography}{99}
\bibitem[Barnes et al. (2006)]{Barnes06}
Barnes, D.G., Fluke, C.J., Bourke, P.D., Parry, O.T., 2006, PASA, 23, 82

\bibitem[Barnes \& Fluke (2008)]{Barnes08}
Barnes, D.G., Fluke, C.J., 2008, New Astronomy, 13, 599 

\bibitem[Barnes et al. (2013)]{Barnes13}
Barnes, D.G., Vidiassov, M., Ruthensteiner, B., Fluke, C.J., Quayley, M., McHenry, C.R., 2013, PLoS ONE, 8, e69446

\bibitem[Fluke \& Barnes ({\em in prep})]{Fluke14}
Fluke, C.J., Barnes, D.G., 2014, in preparation

\end{thebibliography}

\end{document}
