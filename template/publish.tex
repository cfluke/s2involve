%% template.tex
 %
 % Copyright 2006-2012 David G. Barnes, Paul Bourke, Christopher Fluke
 %
 % This file is part of S2PLOT.
 %
 % S2PLOT is free software: you can redistribute it and/or modify it
 % under the terms of the GNU General Public License as published by
 % the Free Software Foundation, either version 3 of the License, or
 % (at your option) any later version.
 %
 % S2PLOT is distributed in the hope that it will be useful, but
 % WITHOUT ANY WARRANTY; without even the implied warranty of
 % MERCHANTABILITY or FITNESS FOR A PARTICULAR PURPOSE.  See the GNU
 % General Public License for more details.
 %
 % You should have received a copy of the GNU General Public License
 % along with S2PLOT.  If not, see <http://www.gnu.org/licenses/>. 
 %
 % We would appreciate it if research outcomes using S2PLOT would
 % provide the following acknowledgement:
 %
 % "Three-dimensional visualisation was conducted with the S2PLOT
 % progamming library"
 %
 % and a reference to
 %
 % D.G.Barnes, C.J.Fluke, P.D.Bourke & O.T.Parry, 2006, Publications
 % of the Astronomical Society of Australia, 23(2), 82-93.
 %
 % $Id: s2direct.tex,v 1.7 2011/05/19 00:50:36 dbarnes Exp $
 %

\documentclass{article}
\usepackage[margin=1.5cm]{geometry}
\usepackage{graphicx}
\usepackage{hyperref}
\usepackage[3D]{movie15}
\parindent=0pt
\parskip=1.2\baselineskip

\begin{document}

\begin{figure}
\begin{center}
  \vspace{0.12in}
  \includemovie[label=fig:OBJNAME,
    poster=OBJNAME-poster.png,
    3Dcoo=0.007216811180114746 -0.04447293281555176 -0.0332721471786499,
    3Droo=6.881564140319824,
    3Dviews2=s2views.txt,
    3Djscript=s2plot-prc.js,
    controls,
    3Dlights=CAD,
    3Dbg=0 0 0,
    autoplay,
    3Drender=Solid,
  ]{WIDTH\linewidth}{HEIGHT\linewidth}{s2direct.prc}\\
\vspace{0.3cm}
\includegraphics[width=CWDTH\linewidth,height=CHGHT\linewidth,angle=0]{{OBJNAME-cbar}.png}
\end{center}
\caption{OBJNAME. Three-dimensional visualisation performed with the 
{\tt S2PLOT} programming library (Barnes et al. 2006), using {\tt s2fits} 
(Fluke \& Barnes, {\em in prep}).  (Upper panel) This is an interactive 
figure for viewing with Adobe Reader (Version 8.0 or higher).  
(Lower panel) Histogram of $\log_{10}({\rm Intensity})$, scaled to maximum bin value in 
arbitrary units. Data range (top row of numbers) and plotted data range (bottom row of numbers) are 
shown in the image.  Additional data is contained in Table \ref{tbl:meta}. The black `T' shows the 
histogram mean and one standard deviation limits. }
\end{figure}

\begin{figure}
\begin{center}
\includegraphics[width=WIDTH\linewidth,height=HEIGHT\linewidth,angle=0]{{OBJNAME-poster}.png}

\vspace{0.3cm}

\includegraphics[width=CWDTH\linewidth,height=CHGHT\linewidth,angle=0]{{OBJNAME-cbar}.png}
\end{center}
\caption{OBJNAME. Three-dimensional visualisation performed with the {\tt S2PLOT}
programming library (Barnes et al. 2006), using {\tt s2fits} (Fluke \& Barnes, {\em in prep}).  (Upper panel) Printable image of spectral data cube using 3D texture volume rendering. 
(Lower panel) Histogram of $\log_{10}({\rm Intensity})$, scaled to maximum bin value in 
arbitrary units. Data range (top row of numbers) and plotted data range (bottom row of numbers) are 
shown in the image.  Additional data is contained in Table \ref{tbl:meta}. The black `T' shows the 
histogram mean and one standard deviation limits. }
\end{figure}


\section{Metadata}

\begin{table}[h]
\caption{{OBJNAME} meta data from {\tt s2fits}.}
\label{tbl:meta}
\begin{center}
\begin{tabular}{lrl}
{\bf Parameter} & {\bf Value} & {\bf Description}\\
\hline 
\input{OBJNAME-meta.txt}
\hline
\end{tabular}
\end{center}
\end{table}

\section*{Credits}
If you find this approach to visualising spectral data cubes to be useful, 
it would be greatly appreciated if you can cite our work as follows:
{\em Three-dimensional visualisation performed with the {\tt S2PLOT} programming 
library (Barnes et al. 2006), using {\tt s2fits} (Fluke \& Barnes, {\em in prep}).}

Additional references may be made to our original description of the 3D-PDF approach
for astronomy publications (Barnes \& Fluke 2008) and the free, open-source solution 
using the PRC format in \LaTeX (Barnes et al. 2013). 


\begin{thebibliography}{99}
\bibitem[Barnes et al. (2006)]{Barnes06}
Barnes, D.G., Fluke, C.J., Bourke, P.D., Parry, O.T., 2006, PASA, 23, 82

\bibitem[Barnes \& Fluke (2008)]{Barnes08}
Barnes, D.G., Fluke, C.J., 2008, New Astronomy, 13, 599 

\bibitem[Barnes et al. (2013)]{Barnes13}
Barnes, D.G., Vidiassov, M., Ruthensteiner, B., Fluke, C.J., Quayley, M., McHenry, C.R., 2013, PLoS ONE, 8, e69446

\bibitem[Fluke \& Barnes, {\em in prep}]{Fluke14}
Fluke, C.J., Barnes, D.G., 2014, in preparation

\end{thebibliography}

\end{document}
